\def\BLLS{生物再生生命保障系统}
\chapter{设计概况}
\label{chp:intro:begin}

\section{生物再生生命保障系统}

生物再生生命保障系统(Bioregenerative Life Support System,BLSS)是基于生态系统原理将生物技术与工程控制技术有机结合,构建由植物、动物、微生物组成的人工生态系统。水和食物这些人类生活所必需的物质可在系统内循环再生,并为乘员提供类似于地球生物圈的生态环境。人进入这个人工生态系统中,成为生态系统的消费者链环同时发挥控制者的功能,构成闭合人工生态系统(Man-made Closed Ecological Systems, MCES)。

生命保障,是载人航天的一项关键技术,宇航员离开地球,在遥远的太空中生存,离不开空气、水和食物。在我国神舟系列飞船、国际空间站、苏联/俄罗斯和平号空间站中,通常会携带全部物资,或通过物理化学方式再生氧气和水,而宇航员吃的食物只能一次性携带充足,不能再生。然而,如果人类在不远的未来,进行更行长时间、更远距离的太空探索,例如构建月球、火星基地,由于路途遥远,食物完全通过携带储存供给,或进行地面定期补给将变得十分昂贵且很难实现。因此,仅仅依靠携带或物理化学再生方式满足生命保障需求,载人深空探索几乎不可能实现。解决办法是依靠“生物再生”的方式,在月球、火星基地,或是飞向火星的飞船中,构建一个类似地球生物圈的小型生态系统。科学家们把这样一个小型生态系统称为“生物再生生命保障系统”。

生物再生生命保障系统利用高等植物(如作物)和动物来生产食物、利用微生物处理废物,同时再生空气和水,为航天员生命活动提供物质保障的独立、完整、复杂的人工生态系统。它引入了生物技术和生态平衡理念,将工程控制技术和生物技术相结合创造出人工小型生态环境,实现在一定的密闭空间内人和其他生物之间氧气、水分和有机物的再生与循环利用,从而大大减少长期空间活动的地面补给,降低运行成本。另一方面,生物再生式生命保障系统能够控制舱内大气成分、温度、湿度以及压力等环境参数,控制舱内环境污染和有害微生物的繁殖,保持居住环境长期健康、稳定,为航天员创造一个舒适和安全的生活环境。

\section{国内外研究现状}
由于\BLLS 对深空探测的重要性,许多国家早在20世纪70年代就展开了BLSS有人系统构建和有人实验。70年代 俄罗斯建立了世界上第一个成功的BLLS —— BIOS-3,该系统实现了水和氧气基本完全的再生,以及系统内独立控制生态系统的能力 。80年代,美国进行了著名的“生物圈二号”实验(BIOSPHERE-2试验),但是以氧气浓度下降、作物大面积歉收而失败。其后进行的实验还有欧洲的MELiSSA系统和俄罗斯的火星500天项目,均取得了不同程度的成功。

我国在生物再生生命保障系统上取得了长足进步。由北京航空航天大学刘红教授团队研发的地基综合实验系统“月宫一号”于2014年1--5月成功进行了我国首次长期高闭合度集成实验,密闭实验持续了105天,成功完成我国首次长期多人密闭试验,为我国的深空探测的生命保障系统提供了理论支持。

\section{设计任务}
本文提出了一种基于火星地表环境的\BLLS 的一期工程设计方案,称为“火星宫”。
火星功能的主要功能是使人们能够在火星上生活,开展火星生命科学研究以及火星资源勘探和开发工作。
“火星宫”一期可容纳10人,主要开展火星生命科学研究。

\section{火星概况}
\label{chp:intro:end}

火星是太阳系由内往外数的第四颗行星。火星直径约是地球的一半,体积为15\%,质量为11\%,表面积相当于地球陆地面积,密度则比其他三颗类地行星(地球、金星、水星)还要小很多。 以半径、质量、表面重力来说,火星约介于地球和月球中间:火星直径约为月球的两倍、地球的一半;质量约为月球九倍、地球的1/9,表面重力约为月球的2.5倍、地球的2/5。

火星基本上是沙漠行星,地表沙丘、砾石遍布且没有稳定的液态水体。二氧化碳为主的大气既稀薄又寒冷,沙尘悬浮其中,每年常有尘暴发生。火星两极皆有水冰与干冰组成的极冠会随着季节消长。

火星的大气密度只有地球的大约1\%,非常干燥,温度低,表面平均温度零下55 \si{\degreeCelsius},水和二氧化碳易冻结。由于缺少地球的板块运动,火星无法使二氧化碳再次循环到它的大气中,从而无法产生意义重大的温室效应。

与地球相比,火星地质活动较不活跃,地表地貌大部份于远古较活跃的时期形成,有密布的陨石坑、火山与峡谷,包括太阳系最高的山:奥林帕斯山和最大的峡谷:水手号峡谷。另一个独特的地形特征是南北半球的明显差别:南方是古老、充满陨石坑的高地,北方则是较年轻的平原。
