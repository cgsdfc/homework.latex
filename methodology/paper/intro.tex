\section{介绍}
在自然语言处理(NLP)中,词的分布式表示是指为词汇表(vocabulary)里的每一个词都指派一个$n$维的实向量。
近年来,随着模型的不断提出和改进,这些词向量捕获了细粒度的语法和语义规律,
可以显著的改进和简化许多NLP应用,例如:
信息检索(information retrieve)\cite{DBLP:books/daglib/0021593},
文档分类(document classification)\cite{Sebastiani:2002:MLA:505282.505283},
问答系统(question answering)\cite{DBLP:conf/sigir/TellexKLFM03},
命名实体识别(named entity recognition)\cite{DBLP:journals/jmlr/CollobertWBKKK11},
解析(parsing)\cite{DBLP:conf/icml/SocherLNM11},
词性标记(part-of-speech tagging) \cite{DBLP:journals/jmlr/CollobertWBKKK11},
依赖解析(dependency parsing)\cite{DBLP:conf/emnlp/ChenM14} \cite{DBLP:conf/emnlp/KongSSBDS14},
以及机器翻译(machine translation)
\cite{DBLP:conf/acl/LiuYLZ14}
\cite{DBLP:conf/emnlp/KalchbrennerB13}
\cite{DBLP:conf/acl/DevlinZHLSM14}
\cite{DBLP:conf/nips/SutskeverVL14}。
具体的,它们可作为应用的特征\cite{DBLP:conf/acl/TurianRB10} 或者用于初始化神经网络
\cite{NIPS2012_4610}
\cite{DBLP:journals/jmlr/ErhanBCMVB10}
\cite{DBLP:conf/emnlp/GuoCWL14}。
使用各种模型的词向量在不同的文集上训练,一些还被保存起来,用于将来的研究和比较
\cite{DBLP:journals/corr/abs-1301-3781}。

为了理解不同的词向量模型是如何捕获语法和语义规律的,本文对不同的模型进行了分类,分析和比较。
本文的内容如下:
\cref{sec:his-of-dist-rep} 介绍了词的分布式表示的历史;
\cref{sec:mea-syn-and-sem-reg} 介绍了衡量词向量的语法和语义规律的向量偏移方法(vector offset method);
\cref{sec:diff-models} 对不同的模型进行了分类,分析和比较,讨论了它们的优点和局限性;
\cref{sec:diss-and-concl} 总结了本文并给出了词向量模型几个可能的发展方向。
