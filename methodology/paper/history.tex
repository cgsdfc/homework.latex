\section{词的分布式表示的历史}
\label{sec:his-of-dist-rep}

\subsection{分布式假设}

分别式假设是在上个世纪70年代由美国语言学家Harris Zellig提出一系列
对语言本质的假设\cite{harris54},其主要内容是:
具有相似分布规律的语言实体也具有相似的意思;或者,具有相似意思的语言实体通常出现在相似的位置。
基于这个假设的方法通常利用语言的分布式信息,例如,单词出现在彼此的上下文中,
或单词有着相似的上下文,来获取语言实体的含义\cite{Sahlgren2008}。

用基于分布式假设获取的含义称为分布式含义。
这种含义和客观世界的对象(例如汽车)或者主观世界的思想(例如相对论)并没有直接关系\cite{Sahlgren2008}。
分布式含义完全建立在语言实体的区别的基础上,换句话说, 分布式含义的本质是语言实体的区别。% TODO 法国人的引用
这个思想为基于向量偏移的词向量相似度评估方法奠定了基础。

\subsection{词的分布式表示}
词的分布式表示有着漫长的历史,早期的提出者包括
\cite{hinton:learndistrep} % 亨顿
\cite{DBLP:journals/ai/Pollack90} % 伯拉
\cite{DBLP:journals/jasis/DeerwesterDLFH90} % 迪恩韦斯特
等人。
Bengio等人在\cite{DBLP:journals/jmlr/BengioDVJ03}\cite{Bengio2006}中
提出了神经网络语言模型(NNLM),又称神经概率语言模型(NPLM)。
该模型在训练一个语言的概率分布函数的同时,在投影层获得了词向量。
该模型带有非线性的隐藏层,并且对整个词汇表进行了softmax,这些复杂度较高的操作限制了它的扩展性。
后人分别在softmax和隐藏层上进行了改进:
\cite{DBLP:conf/nips/MnihH08}和\cite{DBLP:conf/aistats/MorinB05}提出了分层softmax
(hierarchical softmax),成功的提高了模型的效率;
Mikolov等人在\cite{4960686}和\cite{DBLP:journals/corr/abs-1301-3781}均使用了没有隐藏层的简单模型,提升了模型的整体效率;
Mikolov等人还使用循环神经网络取代NNLM的前馈神经网络训练词向量,
以获得某种短期记忆
\cite{DBLP:conf/interspeech/MikolovKBCK10}
\cite{DBLP:conf/naacl/MikolovYZ13}。

Mikolov等人在\cite{DBLP:journals/corr/abs-1301-3781}中提出了两种训练词向量的高效模型:
连续词袋模型(Continuous Bag-of-Word,CBOW)和Skip-gram模型,二者又称为\texttt{word2vec}模型。
它们的基本原理分别是给定一个上下文窗口内的词
预测窗口中心的词(CBOW)和给定中心词预测上下文词(Skip-gram),
并根据预测结果最大化模型的对数似然度(log likehood)。
这些模型能在十亿数量级的文本上进行高效的训练,而且得到的词向量具有丰富的语法和语义
规律。此后,人们提出了许多这两个模型的扩展:
\cite{DBLP:conf/emnlp/LingTAFDBTL15}为CBOW引入了把关注度机制,克服了CBOW对词序不敏感
% 注意力 关注度
的问题,提高了在语法任务上的准确率;
\cite{DBLP:journals/tacl/BojanowskiGJM17}为CBOW引入了词法信息;
\cite{DBLP:conf/acl/SunGLXC15}则受到\cite{Sahlgren2008}的启发,
把语段关系(syntagmatic)和范式关系(paradigmatic)同时引进了CBOW和Skip-gram,
得到了两个新的模型,提高了在语法和语义任务上的性能。

另一方面,Pennington等人提出了Glo Vec模型\cite{pennington2014glove}。与\texttt{word2vec}的模型不同的
是,该模型直接在全局词--词共现矩阵上训练了一个加权最小二乘目标函数。该模型在多项任务中均超过了
\texttt{word2vec},而且更高效的利用了文集的统计信息。


